En este programa modular se ofrece una simulación de la gestión deportiva y cálculo del ranking de un circuito de torneos de tenis. Se introducen las clases {\itshape \mbox{\hyperlink{class_jugador}{Jugador}}}, {\itshape \mbox{\hyperlink{class_torneo}{Torneo}}}, {\itshape \mbox{\hyperlink{class_lista___jugadores}{Lista\+\_\+\+Jugadores}}}, {\itshape \mbox{\hyperlink{class_lista___torneos}{Lista\+\_\+\+Torneos}}} y {\itshape \mbox{\hyperlink{class_categorias}{Categorias}}}.

Este programa concede una serie de instrucciones\+:
\begin{DoxyItemize}
\item Primero de todo, se leerán las categorías, torneos y jugadores iniciales antes de empezar a usar el programa
\item Después, dispones de una serie de operaciones las cuales puedes acceder mediantes la siguientes funciones por consola\+:
\begin{DoxyItemize}
\item Nuevo\+\_\+\+Jugador (nuevo\+\_\+jugador o nj)\+: añade un jugador a la lista de jugadores iniciada previamente. A este se le tiene que añadir un nombre. Si este ya está existe, saltará un error.
\item Nuevo\+\_\+\+Torneo (nuevo\+\_\+torneo o nt)\+: añade un torneo a la lista de torneos iniciada previamentemente. A este se le tiene que añadir un nombre y una categoría. Si este nombre ya existe o la categoría no está definida, saltará error.
\item Baja\+\_\+\+Jugador (baja\+\_\+jugador o bj)\+: da de baja un jugador y actualiza la puntuación de este en todos los torneos donde haya estado inscrito. A este se le tiene que añadir el identificador del juagdor el cual se quiere remover. Si el identificador del jugador no existe, saltará error.
\item Baja\+\_\+\+Torneo (baja\+\_\+torneo o bt)\+: da de baja un torneo, actualizando todos los puntos de los jugadores que hayan estado inscritos en este torneo. A este se le tiene que añadir el identificador del torneo el cual se quiere remover. Si el idenficador del torneo no existe, saltará error.
\item Inicar\+\_\+\+Torneo (iniciar\+\_\+torneo o it)\+: inicia un torneo y imprime el los emparejamientos de este. Primero se tiene que añadir el nombre del torneo que se quiere añadir, seguido del numero de jugadores \char`\"{}n\char`\"{} que se quiere añadir y por último el ranking de los \char`\"{}n\char`\"{} jugadores en orden creciente.
\item Finalizar\+\_\+\+Torneo (finalizar\+\_\+torneo o ft)\+: finaliza un torneo y imprime el cuadro de resultados y los puntos conseguido por lo jugadores inscritos, actualizando los puntos de dichos jugadores, tanto para lo que han jugado como los inscritos anterior mente que no han jugado.
\item Listar\+\_\+\+Rankign (listar\+\_\+ranking o lr)\+: primero te dice cuantos jugadores hay seguido de el ranking en orden creciente del ranking imprimiendo el numero de ranking, el nombre del jugador y los puntos conseguidos.
\item Listar\+\_\+\+Jugadores (listar\+\_\+jugadores o lj)\+: lista los jugadors en orden alfabetico mostrando el ranking, los puntos ,los torneos juagdos, los partidos tanto ganados como perdidos, los setts tanto ganados como perdidos y los juegos tanto ganados como perdidos de cada jugador.
\item Consultar\+\_\+\+Jugador (consultar\+\_\+jugador o cj)\+: lista solo 1 jugador imprimiendo las mismas características de \char`\"{}\+Listar Jugadores\char`\"{} pero solo con un jugador. A este se le tiene que añadir el nombre del jugador.
\item Listar\+\_\+\+Torneos (listar\+\_\+torneos o lt)\+: primero te dice cuantos torneos hay seguido de los torneos imprimiendo los nombres de los torneos con su respectivos nombres de las categorías a las que estan asociadas cada torneo.
\item Por último (fin) hace que finalice el programa. 
\end{DoxyItemize}
\end{DoxyItemize}